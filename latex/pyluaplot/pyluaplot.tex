\documentclass[10pt, twocolumn]{article}
\author{Alessandro Rizzoni}
\title{PyLuaPlot}
\date{\today}

% Localization
\usepackage{polyglossia}
\usepackage{csquotes}
\setdefaultlanguage[variant=american]{english}

% Math and Fonts
\usepackage{mathtools}
\usepackage[warnings-off={mathtools-colon,mathtools-overbracket}]{unicode-math}
\usepackage{lualatex-math}

\usepackage[stixtwo]{fontsetup}
\setsansfont{Courier Prime Sans}[%
    Ligatures=TeX,
    Scale=MatchLowercase
    ]
\setmonofont{Courier Prime Code}[%
    Ligatures=TeX,
    Scale=MatchLowercase
    ]

% Matplotlib - PyLuaTex Interface
\usepackage{pyluaplot}

% Code Typesetting
\usepackage{listings}
\lstset{%
  basicstyle=\small\ttfamily,
  commentstyle=\itshape,
  keywordstyle=\bfseries,
  }

\begin{document}

\maketitle

\begin{abstract}

  This class is a writing environment designed to integrate LuaLaTeX and Python
  3 to create reproducible documents with executable and typesettable code,
  outputs, and errors. This document class may be useful to scientists,
  engineers, researchers, or anyone who needs to create a reproducible document
  with significant plotting, computation, or other needs. This project is at an
  early stage and has not been versioned yet, so consider it unsupported and use
  at your own risk.

  If you want to try this, you \emph{must} declare a math font. I prefer to use
  the wonderful \texttt{fontsetup} package. Please read the code --- this is
  pretty close to a minimum viable implementation (with the notable exception of
  the python file --- that has a few extra bells and whistles).

\end{abstract}

\section{Some Examples in Brief}

\subsection{Example 1}

As a first example, Figure \ref{fig:sine} shows a sinusoidal waveform and the
code used to generate it is shown above. Note that there are no additional
external files generated using this method. The Python utilities provided with
the class define some utility classes that can bring certain information from
the LaTeX context to the Python context. Matplotlib is used as the plotting
engine.

\begin{figure}[!ht]
  \begin{pycode}
    import numpy as np

    time = np.linspace(0, 1, 1000)
    frequency = 10
    omega = 2 * np.pi * frequency
    sine = np.sin(omega * time)

    fig, ax = plt.subplots(1, 1)
    ax.plot(time, sine, label="Sine Wave")
    ax.set_xlabel(r"Time $\left[\mathrm{s}\right]$")
    ax.set_ylabel(r"Voltage $\left[\mathrm{V}\right]$")
    ax.xaxis.set_major_formatter(plot.get_eng_formatter(r"s"))
    ax.yaxis.set_major_formatter(plot.get_eng_formatter(r"V"))

    plot.output_figure(fig)
  \end{pycode}
  \caption{A sine wave}\label{fig:sine}
\end{figure}

\subsection{Example 2}

For the second example, we look at a similar plot of a damped sinusoid. We can
reuse the sine vector that we created in the previous example because every
\texttt{pycode} environment shares the same interpreter thanks to
PyLuaTex. The code used to generate the plot is below.

\begin{figure}[!ht]
  \begin{pycode}
    time = np.linspace(0, 1, 1000)
    damped_sine = sine * np.exp(-5 * time)
    damped_sine = damped_sine / np.max(damped_sine)

    fig, ax = plt.subplots(1, 1)
    ax.plot(time, damped_sine, label="Damped Sine Wave")
    ax.set_xlabel(r"Time $\left[\mathrm{s}\right]$")
    ax.set_ylabel(r"Voltage $\left[\mathrm{V}\right]$")
    ax.xaxis.set_major_formatter(plot.get_eng_formatter(r"s"))
    ax.yaxis.set_major_formatter(plot.get_eng_formatter(r"V"))

    plot.output_figure(fig)
  \end{pycode}
  \caption{A damped sine wave}
\end{figure}

\pagebreak\onecolumn

\section{The Style File}

\lstinputlisting[language=Tex]{pyluaplot.sty}

\section{The LaTeX File}

\lstinputlisting[language=Tex]{pyluaplot.tex}

\section{The Python File}

\lstinputlisting[language=Python]{pyluaplot.py}

\end{document}
