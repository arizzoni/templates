\documentclass[12pt]{notes}
\renewcommand{\theauthor}{Alessandro Rizzoni}
\renewcommand{\thedoctitle}{Electric Machines Notes}
\author{\theauthor}
\title{\thedoctitle}

\begin{document}

\section{Introduction}

Along with power systems, power electronics, and high voltage engineering, the study of \emph{electric machines} is a core topic of the power field within electrical engineering. Fig. \ref{fig:electric_machines_context} shows where the study of electric machines fits into the larger picture of electrical engineering.

\begin{figure}[!ht] 
    \begin{center}
        \begin{circuitikz}[scale=1, transform shape]
            \node[draw, text width=0.75in, minimum height=0.5in, align=center] (power) at (0,0) {Power};
            \node[draw, text width=0.75in, minimum height=0.5in, align=center] (powersystems) at (0,2) {Power Systems};
            \node[draw, text width=0.75in, minimum height=0.5in, align=center] (powerelectronics) at (3,0) {Power Electronics};
            \node[draw, text width=0.75in, minimum height=0.5in, align=center] (highvoltage) at (0,-2) {High Voltage};
            \node[draw, text width=0.75in, minimum height=0.5in, align=center, fill=gray!25] (electricmachines) at (-3,0) {\emph{Electric Machines}};
            \draw (power)
                to[short] (powersystems);
            \draw (power)
                to[short] (powerelectronics);
            \draw (power)
                to[short] (highvoltage);
            \draw (power)
                to[short] (electricmachines);
        \end{circuitikz}
        \caption{Context of Electric Machines within Electrical Engineering \cite{zhang_lectures}}
    \label{fig:electric_machines_context}
    \end{center}
\end{figure}

\subsection{Modern Electric Machine Drive Systems}
\begin{figure}[!ht] 
    \begin{center}
            \usetikzlibrary{arrows.meta}
        \begin{circuitikz}[scale=1, transform shape]
            \node[draw, text width=0.75in, minimum height=0.5in, align=center] (powersource) at (0,0) {Power Source};
            \node[draw, text width=0.75in, minimum height=0.5in, align=center] (powerelec) at (3,0) {Power Electronic Circuits};
            \node[draw, text width=0.75in, minimum height=0.5in, align=center] (elecmach) at (6,0) {Electric Machine};
            \node[draw, text width=0.75in, minimum height=0.5in, align=center] (controller) at (1.5,-2) {Digital Controller};
            \node[draw, text width=0.75in, minimum height=0.5in, align=center] (sensors) at (4.5,-2) {Sensing System};
            \draw (powersource)
                to[short] (powerelec);
            \draw (powerelec)
                to[short] (elecmach);
            \draw (elecmach)
                to[short] (sensors);
            \draw (powerelec)
                to[short] (controller);
            \draw (sensors)
                to[short] (powerelec);
            \draw (sensors)
                to[short] (controller);
        \end{circuitikz}
        \caption{Block diagram of a modern electric machine drive system \cite{zhang_lectures}}
    \label{fig:electric_machine_drive_block_diagram}
    \end{center}
\end{figure}

\begin{table}[!ht]
    \begin{center}
            \caption{Table accompanying Fig. \ref{fig:electric_machine_drive_block_diagram} \cite{zhang_lecture}}
        \begin{tabular}{ c c }
            \hline
            Power Source & AC Grid or Batteries \\
            Power Electronic Circuits & High efficiency power conversion \\
            Electric Machine & DC Machines or AC Machines \\
            Sensing System & Sensors for feedback control, fault diagnosis, and protection \\
            Digital Controller & Control for power conversion \\
            \hline
        \end{tabular}
        \label{tab:electric_machine_drive_block_diagram}
    \end{center}
\end{table}


\subsection{Typical Electric Machine and Drive System}
MACHINE DRIVE BLOCK DIAGRAM GOES HERE

variable voltage variable frequency (VVVF), variable speed drive (VSD)

This architecture will be the focus of the class.

\section{Review of Magnetic Concepts}
\section{Magnetic Circuits}
\section{Energy Convertion of EM Devices}
\section{DC Machines}
\section{Rotating Magnetomotive Force of AC Machine and Windings}
\section{Reference Frame Transformation}
\section{Mathematical Modeling of Synchronous Machines}
\section{Analysis of Synchronous Machines}
\section{Vector Control of Synchronous Machines}
\section{Control of Synchronous Machines with DC/AC Inverter}
\section{Mathematical Modeling of Induction Machines}
\section{Indirect Field-Oriented Control of Induction Machines}

\printbibliography

\end{document}
