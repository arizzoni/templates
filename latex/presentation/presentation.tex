\documentclass[aspectratio=1609]{presentation}
\title{Presentations with PyDoc}
\subtitle{Never use PowerPoint again!}
\author{Alessandro Rizzoni | \texttt{rizzoni.3@osu.edu}}
\institute{%
  \normalsize The Ohio State University
  \break\small College of Engineering
  \break\footnotesize Dept.\ of Electrical and Computer Engineering
}
\date{\today}

\begin{document}
  
\begin{frame}[plain]\titlepage\end{frame}

\begin{frame}

  \frametitle{Introduction}

  This class is a writing environment designed to integrate {\LuaLaTeX} and
  Python 3 to create reproducible documents with executable and typesettable
  code, outputs, and errors. This may be useful to scientists, engineers,
  researchers, or anyone who needs to create a reproducible document with
  significant plotting, computation, or other needs. The project is at an early
  stage and has not been versioned yet, so consider it unsupported and use at
  your own risk.

  \vspace{\baselineskip}

  That being said, connecting a Python interpreter to some code that someone
  you don't know wrote is fairly risky. Make sure you understand what you are
  doing. You have been warned.

\end{frame}

\begin{frame}[fragile]{The Python Interpreter}
  \framesubtitle{The Python Interface}
  \begin{columns}
    \begin{column}{0.4\textwidth}
      \begin{itemize}
        \item We can import \texttt{NumPy} like normal
        \item Let's initialize some variables as an example:
        \begin{itemize}
          \item A time vector, \texttt{time}
          \item Frequency of oscillation, \texttt{freq}
          \item Angular frequency, \texttt{omega}
        \end{itemize}
        \item And some signals:
        \begin{itemize}
          \item A sinusoid,
            \begin{itemize}
              \item \texttt{sine} $ = \sin\left(2\pi f t\right)$
            \end{itemize}
          \item A derived square wave,
            \begin{itemize}
              \item \texttt{square} $ = \mathrm{sgn}\left(v_1(t)\right)$
            \end{itemize}
        \end{itemize}
      \end{itemize}
    \end{column}
    \begin{column}{0.6\textwidth}
      \begin{pycode}
        import numpy as np

        time = np.linspace(0, 1, 999)
        freq = 10
        omega = 2 * np.pi * freq
        sine = np.sin(omega * time)
        square = np.sign(sine)
      \end{pycode}
      {\pytypeset}
    \end{column}
  \end{columns}
\end{frame}

\begin{frame}[fragile]{A First Plot with PyLuaTex}
  \framesubtitle{An Introduction to Plotting}
  \begin{columns}
    \begin{column}{0.4\textwidth}
      \begin{itemize}
        \item Plot the signals with \texttt{MatPlotLib}:
        \begin{itemize}
          \item A sine wave
          \item A square wave
        \end{itemize}
        \item Contextualized geometry
        \begin{itemize}
          \item Plot scaling is 1:1
          \item Graphics are sized to fit automatically
          \item The same typeface, at the same size
        \end{itemize}
        \item Uses PGF as a backend, so all plots are native {\LaTeX}
      \end{itemize}
    \end{column}
    \begin{column}{0.6\textwidth}
      \begin{pycode}
        fig, ax = plt.subplots(1, 1, sharex=True)
        sf = plot.get_eng_formatter(r"s")
        vf = plot.get_eng_formatter(r"s")
        ax.plot(time, sine, label=r"\texttt{sine}")
        ax.plot(time, square, label=r"\texttt{square}")
        ax.xaxis.set_major_formatter(sf)
        ax.yaxis.set_major_formatter(vf)
        ax.set_ylabel(r"Voltage $\left[\mathrm{V}\right]$")
        ax.set_xlabel(r"Time $\left[\mathrm{s}\right]$")
        ax.legend(ncols=2, bbox_to_anchor=(0.5, 1), loc="lower center")
        plot.output_figure(fig)
      \end{pycode}
    \end{column}
  \end{columns}
\end{frame}

\begin{frame}[fragile]{Making the Plot}
  \begin{columns}
    \begin{column}{0.4\textwidth}
      \begin{itemize}
        \item Plot Generation Code
        \begin{itemize}
          \item Continued line numbers from before --- it's the same session!
          \item Use {\LaTeX} in the plots
          \item \texttt{plot.output\_figure(fig)} function handles
            output
        \end{itemize}
      \end{itemize}
    \end{column}
    \begin{column}{0.6\textwidth}
      {\pytypeset}
    \end{column}
  \end{columns}
\end{frame}

\end{document}
