\documentclass[12pt]{document}
\author{Alessandro Rizzoni}
\title{Document Class}
\date{\today}

\begin{document}

\maketitle

\begin{introduction}

  This class is a writing environment designed to integrate
  {\LuaLaTeX} and Python 3 to create
  reproducible documents with executable and typesettable code, outputs, and
  errors \ref{python}. This document class may be useful to
  scientists, engineers, researchers,
  or anyone who needs to create a reproducible document with
  significant plotting,
  computation, or other needs. This project is at an early stage and
  has not been
  versioned yet, so consider it unsupported and use at your own risk.

\end{introduction}

\section{Some Examples in Brief}

\subsection{Example 1}

As a first example, Figure \ref{fig:sine} shows a sinusoidal waveform
and the code used to generate it is shown above. Note that there are no additional external files
generated using this method. The Python utilities provided with the class define
some utility classes that can bring certain information from the LaTeX context
to the Python context. Matplotlib is used as the plotting engine
\ref{matplotlib}.

\begin{figure}[!ht]
  \begin{pycode}
    import numpy as np

    time = np.linspace(0, 1, 1000)
    frequency = 10
    omega = 2 * np.pi * frequency
    sine = np.sin(omega * time)

    fig, ax = plt.subplots(1, 1)
    ax.plot(time, sine, label="Sine Wave")
    ax.set_xlabel(r"Time $\left[\mathrm{s}\right]$")
    ax.set_ylabel(r"Voltage $\left[\mathrm{V}\right]$")
    ax.xaxis.set_major_formatter(plot.get_eng_formatter(r"s"))
    ax.yaxis.set_major_formatter(plot.get_eng_formatter(r"V"))

    plot.output_figure(fig)
  \end{pycode}
  \caption{A sine wave}\label{fig:sine}
\end{figure}

\pytypeset%

\subsection{Example 2}

For the second example, we look at a similar plot of a damped sinusoid. We can
reuse the sine vector that we created in the previous example because every
\texttt{pycode} environment shares the same interpreter thanks to
PyLuaTex. The code used to generate the plot is below.

\begin{figure}[!ht]
  \begin{pycode}
    time = np.linspace(0, 1, 1000)
    damped_sine = sine * np.exp(-1 * time)
    damped_sine = damped_sine / np.max(damped_sine)

    fig, ax = plt.subplots(1, 1)
    ax.plot(time, damped_sine, label="Damped Sine Wave")
    ax.set_xlabel(r"Time $\left[\mathrm{s}\right]$")
    ax.set_ylabel(r"Voltage $\left[\mathrm{V}\right]$")
    ax.xaxis.set_major_formatter(plot.get_eng_formatter(r"s"))
    ax.yaxis.set_major_formatter(plot.get_eng_formatter(r"V"))

    plot.output_figure(fig)
  \end{pycode}
  \caption{A damped sine wave}\label{fig:dampedsine}
\end{figure}

\pytypeset%

\end{document}
