\documentclass[10pt]{homework}

\author{Homework Template \\ Spring 2024 \\ Alex Rizzoni}
\title{\small The Ohio State University \\ College of Engineering \\ Electrical \& Computer Engineering Dept.}

\begin{document}

\section{Sinusoidal Voltages}

\begin{center}
    \begin{figure}
        \begin{pycode}
import numpy as np

FREQUENCY = 4
OMEGA = 2 * np.pi * FREQUENCY
PHI = np.pi / 4

t = np.linspace(0, 1, 1000)

x_1 = np.cos(OMEGA * t + PHI)
x_2 = np.sin(OMEGA * t + PHI)

fig, ax = plt.subplots(1, 1)
ax.plot(t, x_1, label="$x_1$")
ax.plot(t, x_2, label="$x_2$")
ax.plot(t, -x_1, label="$x_3$")
ax.plot(t, -x_2, label="$x_4$")
ax.set_xlabel(r"Time $\left[\mathrm{s}\right]$")
ax.set_ylabel(r"Voltage $\left[\mathrm{V}\right]$")
ax.xaxis.set_major_formatter(plot.get_eng_formatter(r"s"))
ax.yaxis.set_major_formatter(plot.get_eng_formatter(r"V"))
ax.legend(ncols=len(ax.lines), bbox_to_anchor=(0.5, 1), loc="lower center")
fig.align_ylabels()

plot.output_figure(fig)
        \end{pycode}
        \caption{Sinusoidal Functions}\label{fig_sine_funcs}
    \end{figure}
\end{center}

The plot in Fig. \ref{fig_sine_funcs} shows two sinusoidal signals. They can be described mathematically as shown in Eq. \ref{eq_sine_funcs}.

\begin{equation} \label{eq_sine_funcs}
    \begin{split}
        x_1 &= \cos\left(2\pi ft + \phi\right) \\
            &= \cos\left(2\pi \times \py{FREQUENCY}t + \frac{\pi}{\py{np.round(1/(PHI/np.pi), 1)}}\right) \\
        x_2 &= \sin\left(2\pi ft + \phi\right) \\
            &= \sin\left(2\pi \times \py{FREQUENCY}t + \frac{\pi}{\py{np.round(1/(PHI/np.pi), 1)}}\right) \\
        x_1 &= -\cos\left(2\pi ft + \phi\right) \\
            &= -\cos\left(2\pi \times \py{FREQUENCY}t + \frac{\pi}{\py{np.round(1/(PHI/np.pi), 1)}}\right) \\
        x_4 &= -\sin\left(2\pi ft + \phi\right) \\
            &= -\sin\left(2\pi \times \py{FREQUENCY}t + \frac{\pi}{\py{np.round(1/(PHI/np.pi), 1)}}\right) \\
        \mathrm{for\ } f &= \py{FREQUENCY} \mathrm{\ and\ } \phi = \frac{\pi}{\py{np.round(1/(PHI/np.pi), 1)}}
    \end{split}
\end{equation}

\end{document}
